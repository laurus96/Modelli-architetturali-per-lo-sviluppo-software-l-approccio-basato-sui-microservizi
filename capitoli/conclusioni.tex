\chapter{Conclusioni}
Lo scopo di questo lavoro era quello di presentare il modello architetturale basato sui microservizi. L'avanzamento delle tecnologie ha portato ad una crescente richiesta di software sempre più impegnativi, l'approccio di sviluppo col modello monolitico inizia a non essere più la scelta più ideale.

Abbiamo discusso di alcune delle tecnologie che implementato concetti teorici di inizi anni 2000 e abbiamo presentato un lavoro che basato sull'approccio ai microservizi ha reso evidente i vantaggi offerti da questa architettura. Le tecnologie che negli anni si sono perfezionate hanno migliorato sempre di più l'architettura ai microservizi, ciò che prima poteva creare dei problemi oggi è stato risolto tramite diverse tecnologie.

Buffo, per la creazione di un applicativo a microservizi abbiamo dovuto far affidamento su diverse tecnologie che offrissero diversi servizi e a sua volta ogni strumento utilizzato e stato creato facendo affidamento ad altri servizi. Questa ridondanza è ciò che rende questo approccio così versatile è potete, non importa come un servizio sia stato scritto o implementato.

Con il sempre più crescente interesso verso il cloud computing, le applicazioni e i software stanno cambiando drasticamente.
