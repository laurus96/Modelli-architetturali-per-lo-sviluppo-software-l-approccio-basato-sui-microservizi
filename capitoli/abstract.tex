\chapter*{Abstract}
Il tradizionale approccio allo sviluppo software prevede che tutte le componenti del sistema siano
integrate in un’unica grande unità, tale metodologia di sviluppo è conosciuta col nome di architettura
monolitica. Un software monolitico è logicamente autonomo, in quanto i suoi componenti sono fortemente
accoppiati e interdipendenti.

Negli ultimi anni si è cercato di realizzare un approccio architetturale che non rendesse i componenti di
un sistema fortemente accoppiati, un vantaggio molto prezioso considerando che le applicazioni di oggi
sono sempre più vaste e ricche di funzionalità, l'architettura a microservizi prevede tale vantaggio e non solo.

Un sistema costruito con l'approccio ai microservizi è suddiviso in piccole componenti indipendenti che lavorano insieme per fornire funzionalità complesse. Ciascun microservizio è
responsabile di una specifica parte dell'applicazione e può essere sviluppato, testato e distribuito in
modo indipendente dal resto dell'applicativo.

In contrasto con l'architettura monolitica, l'utilizzo dei microservizi permette di avere maggiore
flessibilità e scalabilità nello sviluppo delle applicazioni. I microservizi rendono più facile
la manutenzione dell'applicazione, poiché è possibile modificare, sostituire o aggiungere un singolo microservizio
senza dover modificare più parti dell'applicazione, riducendo in modo significativo il rischio di creare regressione
all'interno del codice.

Lo scopo di questo lavoro di tesi è quello di esplorare e descrivere diverse tecnologie al fine di riuscire a produrre un'applicazione basata sui microservizi.